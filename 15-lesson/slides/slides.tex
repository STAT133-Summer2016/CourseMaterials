\documentclass[12pt]{beamer}\usepackage[]{graphicx}\usepackage[]{color}
%% maxwidth is the original width if it is less than linewidth
%% otherwise use linewidth (to make sure the graphics do not exceed the margin)
\makeatletter
\def\maxwidth{ %
  \ifdim\Gin@nat@width>\linewidth
    \linewidth
  \else
    \Gin@nat@width
  \fi
}
\makeatother

\definecolor{fgcolor}{rgb}{0.345, 0.345, 0.345}
\newcommand{\hlnum}[1]{\textcolor[rgb]{0.686,0.059,0.569}{#1}}%
\newcommand{\hlstr}[1]{\textcolor[rgb]{0.192,0.494,0.8}{#1}}%
\newcommand{\hlcom}[1]{\textcolor[rgb]{0.678,0.584,0.686}{\textit{#1}}}%
\newcommand{\hlopt}[1]{\textcolor[rgb]{0,0,0}{#1}}%
\newcommand{\hlstd}[1]{\textcolor[rgb]{0.345,0.345,0.345}{#1}}%
\newcommand{\hlkwa}[1]{\textcolor[rgb]{0.161,0.373,0.58}{\textbf{#1}}}%
\newcommand{\hlkwb}[1]{\textcolor[rgb]{0.69,0.353,0.396}{#1}}%
\newcommand{\hlkwc}[1]{\textcolor[rgb]{0.333,0.667,0.333}{#1}}%
\newcommand{\hlkwd}[1]{\textcolor[rgb]{0.737,0.353,0.396}{\textbf{#1}}}%

\usepackage{framed}
\makeatletter
\newenvironment{kframe}{%
 \def\at@end@of@kframe{}%
 \ifinner\ifhmode%
  \def\at@end@of@kframe{\end{minipage}}%
  \begin{minipage}{\columnwidth}%
 \fi\fi%
 \def\FrameCommand##1{\hskip\@totalleftmargin \hskip-\fboxsep
 \colorbox{shadecolor}{##1}\hskip-\fboxsep
     % There is no \\@totalrightmargin, so:
     \hskip-\linewidth \hskip-\@totalleftmargin \hskip\columnwidth}%
 \MakeFramed {\advance\hsize-\width
   \@totalleftmargin\z@ \linewidth\hsize
   \@setminipage}}%
 {\par\unskip\endMakeFramed%
 \at@end@of@kframe}
\makeatother

\definecolor{shadecolor}{rgb}{.97, .97, .97}
\definecolor{messagecolor}{rgb}{0, 0, 0}
\definecolor{warningcolor}{rgb}{1, 0, 1}
\definecolor{errorcolor}{rgb}{1, 0, 0}
\newenvironment{knitrout}{}{} % an empty environment to be redefined in TeX

\usepackage{alltt}
\usepackage{graphicx}
\usepackage{tikz}
\setbeameroption{hide notes}
\setbeamertemplate{note page}[plain]
\usepackage{listings}

\input{../../theme/header.tex}

%------------------------------------------------
% end of header
%------------------------------------------------

\title{Lesson 15 - XML}
\subtitle{STAT 133}
\author{Andrew Do}
\institute{Department of Statistics, UC{\textendash}Berkeley}
\date{\scriptsize \color{foreground} Adapted from Gaston Sanchez's Fall 2015 STAT133 lectures.
\\
\href{http://github.com/gastonstat}{\tt \scriptsize \color{foreground} github.com/gastonstat}
\\
{\scriptsize Course web: \href{https://github.com/STAT133-Summer2016/CourseMaterials}{\tt github.com/STAT133-Summer2016/CourseMaterials}}
}
\IfFileExists{upquote.sty}{\usepackage{upquote}}{}
\begin{document}


{
  \setbeamertemplate{footline}{} % no page number here
  \frame{
    \titlepage
  } 
}

%------------------------------------------------

\begin{frame}
\begin{center}
\humongous{\hilit{XML}}
\end{center}
\end{frame}

%------------------------------------------------

\begin{frame}
\begin{center}

\bb{XML \& HTML}
This lecture will be a \textbf{crash introduction to XML and HTML}.  You'll get a gist of how the two languages work so that you can do basic manipulations of data written in these languages, but \textbf{this will be far from comprehensive}.
\eb

\end{center}
\end{frame}

%------------------------------------------------

\begin{frame}
\frametitle{Datasets}

\bb{So far we've been working with tabular data}
\eb
\begin{center}
\ig[width=10cm]{images/tabular_nontabular.pdf}
\end{center}

\end{frame}

%------------------------------------------------

\begin{frame}

\end{frame}

%------------------------------------------------

\begin{frame}
\frametitle{Motivation}

\bb{Two main limitations of field-delimited files}
\bbi
  \item In plain text formats there is no information to describe the location of the data values
  \item There is no recognizable label for each data value within the file
  \item Serious limitations to store data with hierarchical structure 
\ei
\eb

\end{frame}

%------------------------------------------------

\begin{frame}
\frametitle{Hierarchical data}
\begin{center}
\ig[width=10cm]{images/hierarchical_data.pdf}
\end{center}
\end{frame}

%------------------------------------------------

\begin{frame}[fragile]
\frametitle{Hierarchical data}

Field-delimited files have limitations with hierarchical data
\bigskip

\begin{verbatim}
                   John      33  male
                   Julia     32  female
    John   Julia   Jack       6  male
    John   Julia   Jill       4  female
    John   Julia   John jnr   2  male
                   David     45  male
                   Debbie    42  female
    David  Debbie  Donald    16  male
    David  Debbie  Dianne    12  female
\end{verbatim}

\end{frame}

%------------------------------------------------

\begin{frame}
\frametitle{XML format}

\bb{XML advantages}
\bbi
  \item XML is a storage format that is still based on plain text
  \item In XML formats every single value is distinctly labeled
  \item Moreover, every single value is self-described
  \item The information is organized in a much more sophisticated manner
\ei
\eb

\end{frame}

%------------------------------------------------

\begin{frame}[fragile]
\frametitle{Hierarchical data}

An example of hierarchical data in XML
\begin{knitrout}\footnotesize
\definecolor{shadecolor}{rgb}{0.969, 0.969, 0.969}\color{fgcolor}\begin{kframe}
\begin{alltt}
<family>
  <parent gender=\hlstr{"male"} name=\hlstr{"John"} age=\hlstr{"33"} />
  <parent gender=\hlstr{"female"} name=\hlstr{"Julia"} age=\hlstr{"32"} />
  <child gender=\hlstr{"male"} name=\hlstr{"Jack"} age=\hlstr{"6"} />
  <child gender=\hlstr{"female"} name=\hlstr{"Jill"} age=\hlstr{"4"} />
  <child gender=\hlstr{"male"} name=\hlstr{"John jnr"} age=\hlstr{"2"} />
</family>
<family>
  <parent gender=\hlstr{"male"} name=\hlstr{"David"} age=\hlstr{"45"} />
  <parent gender=\hlstr{"female"} name=\hlstr{"Debbie"} age=\hlstr{"42"} />
  <child gender=\hlstr{"male"} name=\hlstr{"Donald"} age=\hlstr{"16"} />
  <child gender=\hlstr{"female"} name=\hlstr{"Dianne"} age=\hlstr{"12"} />
</family>
\end{alltt}
\end{kframe}
\end{knitrout}

\end{frame}

%------------------------------------------------

\begin{frame}
\frametitle{XML and HTML}

\bb{Why should you care about XML and HTML?}
\bbi
  \item Large amounts of data and information are stored, shared and distributed using HTML and XML-dialects
  \item They are widely adopted and used in many applications
  \item Working with data from the Web means dealing with HTML
\ei
\eb

\end{frame}

%------------------------------------------------

\begin{frame}
 \begin{center}
  {\Huge \hilit XML}
  
  \bigskip
  {\Large \hilit eXtensible Markup Language}
 \end{center}
\end{frame}

%------------------------------------------------

\begin{frame}
\frametitle{Some Definitions}

\begin{quotation}
``XML is a markup language that defines a set of rules for encoding documents in a format that is both human-readable and machine-readable''
\end{quotation}

{\footnotesize 
\hspace{8mm} \url{http://en.wikipedia.org/wiki/XML}
}

\bigskip
\begin{quotation}
``XML is a data description language used for describing data''
\end{quotation}

{\footnotesize 
\hspace{8mm} {\hilit Paul Murrell} \\
\hspace{8mm} {\lolit Introduction to Data Technologies}
}

\end{frame}

%------------------------------------------------

\begin{frame}
\frametitle{Some Definitions}

\begin{quotation}
``XML is a very general structure with which we can define any number of new formats to represent arbitrary data''
\end{quotation}

\begin{quotation}
``XML is a standard for the semantic, hierarchical representation of data''
\end{quotation}

{\footnotesize 
\hspace{8mm} {\hilit Deb Nolan \& Duncan Temple Lang} \\
\hspace{8mm} {\lolit XML and Web Technologies for Data Sciences with R}
}

\end{frame}

%------------------------------------------------

\begin{frame}
\frametitle{About XML}

\bb{XML}
XML stands for \textbf{eXtensible Markup Language}
\eb

\bb{Broadly speaking ...}
XML provides a flexible framework to create formats for describing and representing data
\eb

\end{frame}

%------------------------------------------------

\begin{frame}
\frametitle{Markups}

\bb{Markup}
A \textbf{markup} is a sequence of characters or other symbols inserted at certain places in a document to indicate either: 
\bi
 \item how the content should be displayed when printed or in screen
 \item describe the document's structure
\ei
\eb

\end{frame}

%------------------------------------------------

\begin{frame}
\frametitle{Markups}

\bb{Markup Language}
A markup language is a system for \textbf{annotating} (i.e. \textit{marking}) a document in a way that the content is distinguished from its representation (eg LaTeX, PostScript, HTML, SVG)
\eb

\end{frame}

%------------------------------------------------

\begin{frame}[fragile]
\frametitle{LaTeX example}

\begin{columns}[t]
\begin{column}{0.1\textwidth}
\end{column}
\begin{column}{0.6\textwidth}
\begin{knitrout}\tiny
\definecolor{shadecolor}{rgb}{0.969, 0.969, 0.969}\color{fgcolor}\begin{kframe}
\begin{alltt}
\textbackslash{}documentclass\{article\}
\textbackslash{}usepackage\{graphicx\}

\textbackslash{}begin\{document\}

\textbackslash{}title\{Introduction to XML\}
\textbackslash{}author\{First Last\}
\textbackslash{}maketitle

\textbackslash{}section\{Introduction\}
Here is the text of your introduction.

\textbackslash{}begin\{equation\}
    \textbackslash{}label\{simple_equation\}
    \textbackslash{}alpha = \textbackslash{}sqrt\{ \textbackslash{}beta \}
\textbackslash{}end\{equation\}

\textbackslash{}subsection\{Subsection Heading Here\}
Write your subsection text here.

\textbackslash{}begin\{figure\}
    \textbackslash{}centering
    \textbackslash{}includegraphics[width=3.0in]\{myfigure\}
    \textbackslash{}caption\{Simulation Results\}
    \textbackslash{}label\{simulationfigure\}
\textbackslash{}end\{figure\}

\textbackslash{}end\{document\}
\end{alltt}
\end{kframe}
\end{knitrout}
\end{column}

\begin{column}{0.1\textwidth}
\end{column}

\end{columns}

\end{frame}

%------------------------------------------------

\begin{frame}[fragile]
\frametitle{Markups}

\bb{XML Markups}
In XML (as well as in HTML) the marks (aka \textit{tags}) are defined using angle brackets: {\Large {\hilit \code{<>}}}
\eb

\bigskip

\code{{\hilit <mark>}Text marked with special tag{\hilit </mark>} }

\end{frame}

%------------------------------------------------

\begin{frame}[fragile]
\frametitle{Extensible}

\bb{Extensible?}
The concept of \textit{extensibility} means that we can define our own marks, the order in which they occur, and how they should be processed. For example:
 \bi
  \item \code{<my\_mark>}
  \item \code{<awesome>}
  \item \code{<boring>}
  \item \code{<cool>}
 \ei
\eb

\end{frame}

%------------------------------------------------

\begin{frame}
\frametitle{About XML}

\bb{XML is NOT}
\bbi
 \item a programming language
 \item a network transfer protocol
 \item a database
\ei
\eb

\end{frame}

%------------------------------------------------

\begin{frame}
\frametitle{About XML}

\bb{XML is}
\bbi
 \item more than a markup language
 \item a generic language that provides structure and syntax for representing any type of information
 \item a meta-language: it allows us to create or define other languages
\ei
\eb

\end{frame}

%------------------------------------------------

\begin{frame}
\frametitle{XML Applications}

\bb{Some XML dialects}
\bbi
 \item \textbf{KML} (\textit{Keyhole Markup Language}) for describing geo-spatial information used in Google Earth, Google Maps, Google Sky
 \item \textbf{SVG} (\textit{Scalable Vector Graphics}) for visual graphical displays of two-dimensional graphics with support for interactivity and animation
 \item \textbf{PMML} (\textit{Predictive Model Markup Language}) for describing and exchanging models produced by data mining and machine learning algorithms
\ei
\eb

\end{frame}

%------------------------------------------------

\begin{frame}[fragile]
\frametitle{Keyhole Markup Language example}

\begin{knitrout}\footnotesize
\definecolor{shadecolor}{rgb}{0.969, 0.969, 0.969}\color{fgcolor}\begin{kframe}
\begin{alltt}
<?xml version=\hlstr{"1.0"} encoding=\hlstr{"UTF-8"}?>
<kml xmlns=\hlstr{"http://www.opengis.net/kml/2.2"}>
<Document>
<Placemark>
  <name>New York City</name>
  <description>New York City</description>
  <Point>
    <coordinates>-74.006393,40.714172,0</coordinates>
  </Point>
</Placemark>
</Document>
</kml>
\end{alltt}
\end{kframe}
\end{knitrout}

\end{frame}

%------------------------------------------------

\begin{frame}[fragile]
\frametitle{Scalable Vector Graphics example}

\begin{knitrout}\scriptsize
\definecolor{shadecolor}{rgb}{0.969, 0.969, 0.969}\color{fgcolor}\begin{kframe}
\begin{alltt}
<svg width=\hlstr{"100"} height=\hlstr{"100"}>
  <circle cx=\hlstr{"50"} cy=\hlstr{"50"} r=\hlstr{"40"} stroke=\hlstr{"green"} stroke-width=\hlstr{"4"} />
</svg>
  
  
<svg width=\hlstr{"400"} height=\hlstr{"110"}>
  <rect width=\hlstr{"300"} height=\hlstr{"100"} style=\hlstr{"fill:\hlkwd{rgb}(0,0,255)"} />
</svg>
\end{alltt}
\end{kframe}
\end{knitrout}

\end{frame}

%------------------------------------------------

\begin{frame}
\begin{center}
\Huge{\hilit{Minimalist Example}}
\end{center}
\end{frame}

%------------------------------------------------

\begin{frame}
\frametitle{}
\begin{center}
\ig[width=10cm]{images/goodwillhunting.jpg}
\end{center}
\end{frame}

%------------------------------------------------

\begin{frame}[fragile]
\frametitle{XML Example}

\begin{block}{Ultra Simple XML}
{\Large \begin{verbatim}
<movie>
  Good Will Hunting
</movie>
\end{verbatim}
}
\end{block}

\end{frame}

%------------------------------------------------

\begin{frame}[fragile]
\frametitle{XML Example}

\begin{block}{Ultra Simple XML}
\begin{verbatim}
<movie>
  Good Will Hunting
</movie>
\end{verbatim}
\end{block}

\bigskip

\bi
 \item one single element {\textit{movie}}
 \item start-tag: {\hilit \code{<movie>}}
 \item end-tag: {\hilit \code{</movie>}}
 \item content: \code{Good Will Hunting}
\ei

\end{frame}

%------------------------------------------------

\begin{frame}[fragile]
\frametitle{XML Example}

\bb{Ultra Simple XML}
\begin{verbatim}
<movie mins="126" lang="en">
  Good Will Hunting
</movie>
\end{verbatim}
\eb

\bigskip

\bi
 \item xml elements can have \textbf{attributes}
 \item attributes:  \code{\hilit mins} (minutes) and \code{\hilit lang} (language)
 \item attributes are \textit{attached} to the element's start tag
 \item attribute values \textbf{must be quoted!}
\ei

\end{frame}

%------------------------------------------------

\begin{frame}[fragile]
\frametitle{XML Example}

\bb{Minimalist XML}
\begin{verbatim}
<movie mins="126" lang="en">
  <title>Good Will Hunting</title>
  <director>Gus Van Sant</director>
  <year>1998</year>
  <genre>drama</genre>
</movie>
\end{verbatim}
\eb

\bigskip

\bi
 \item an xml element may contain other elements
 \item \textit{movie} contains several elements: \textit{title, director, year, genre}
\ei

\end{frame}

%------------------------------------------------

\begin{frame}[fragile]
\frametitle{XML Example}

\bb{Simple XML}
\begin{verbatim}
<movie mins="126" lang="en">
  <title>Good Will Hunting</title>
  <director>
    <first_name>Gus</first_name>
    <last_name>Van Sant</last_name>
  </director>
  <year>1998</year>
  <genre>drama</genre>
</movie>
\end{verbatim}
\eb

\bigskip

\bi
 \item Now \textit{director} has two child elements: \textit{first\_name} and \textit{last\_name}
\ei

\end{frame}

%------------------------------------------------

\begin{frame}[fragile]
\frametitle{XML Hierarchy Structure}

\begin{block}{Conceptual XML}
\begin{verbatim}
<Root>
  <child_1>...</child_1>
  <child_2>...</child_2>
    <subchild>...</subchild>
  <child_3>...</child_3>
</Root>
\end{verbatim}
\end{block}

\bigskip

\bi
 \item An XML document can be represented with a \textbf{tree structure}
 \item An XML document must have \textbf{one single Root} element
 \item The \code{Root} may contain \code{child} elements
 \item A \code{child} element may contain \code{subchild} elements
\ei

\end{frame}

%------------------------------------------------

\begin{frame}
\frametitle{}
\begin{center}
\ig[width=10cm]{images/xml_movie_tree1.pdf}
\end{center}
\end{frame}

%------------------------------------------------

\begin{frame}
\frametitle{}
\begin{center}
\ig[width=10cm]{images/xml_movie_tree2.pdf}
\end{center}
\end{frame}

%------------------------------------------------

\begin{frame}
\frametitle{Well-Formedness}

\bb{Well-formed XML}
We say that an XML document is \textbf{well-formed} when it obeys the basic syntax rules of XML. Some of those rules are:
\bi
 \item one root element containing the rest of elements
 \item properly nested elements
 \item self-closing tags
 \item attributes appear in start-tags of elements
 \item attribute values must be quoted
 \item element names and attribute names are case sensitive
\ei
\eb

\end{frame}

%------------------------------------------------

\begin{frame}[fragile]
\frametitle{Well-Formedness}

\begin{verbatim}
<movie mins="126" lang="en">
  <title>Good Will Hunting</title>
  <director>
    <first_name>Gus</first_name>
    <last_name>Van Sant</last_name>
  </director>
  <year>1998</year>
  <genre>drama</genre>
</movie>
\end{verbatim}

\end{frame}

%------------------------------------------------

\begin{frame}
\frametitle{Well-Formedness}

\bb{Importance of Well-formed XML}
Not well-formed XML documents produce potentially fatal errors or warnings when parsed.

\bigskip
Documents may be well-formed but not valid. Well-formed just guarantees that the document meets the basic XML structure, not that the content is valid.
\eb

\end{frame}

%------------------------------------------------

\begin{frame}
\begin{center}
\Huge{\hilit{Additional XML Elements}}
\end{center}
\end{frame}

%------------------------------------------------

\begin{frame}[fragile]
\frametitle{Some Additional Elements}

{ \small
\begin{verbatim}
<?xml version="1.0"? encoding="UTF-8" ?>
<![CDATA[ a > 5 & b < 10 ]]>
<?GS print(format = TRUE)>
<!DOCTYPE Movie>
<!-- This is a comment -->
<movie mins="126" lang="en">
  <title>Good Will Hunting</title>
  <director>
    <first_name>Gus</first_name>
    <last_name>Van Sant</last_name>
  </director>
  <year>1998</year>
  <genre>drama</genre>
</movie>
\end{verbatim}
}

\end{frame}

%------------------------------------------------

\begin{frame}
\frametitle{Additional Optional XML Elements}

{\small 
\begin{tabular}{l l}
  \hline
  Markup & Description \\
  \hline
  \code{<?xml >} & XML Declaration \\
  & \textit{identifies content as an XML document} \\
  \code{<?PI >} & Processing Instruction \\
  & \textit{processing instructions passed to application} \code{PI} \\
  \code{<!DOCTYPE >} & Document-type Declaration \\
  & \textit{defines the structure of an XML document} \\
  \code{<![CDATA[ ]]>} & CDATA Character Data \\
  & \textit{anything inside a CDATA is ignored by the parser} \\
  \code{<!--  -->} & Comment \\
  & \textit{for writing comments} \\
  \hline
\end{tabular}
}

\end{frame}

%------------------------------------------------

\begin{frame}
\frametitle{DTD}

\bb{Document-Type Declaration}
The Document-type Declaration identifies the \textbf{type} of the document. The \textit{type} indicates the structure of a \textbf{valid} document: 

\bi
 \item what elements are allowed to be present
 \item how elements can be combined
 \item how elements must be ordered
\ei
\eb

Basically, the DTD specifies what the format allows you to do.
\end{frame}

%------------------------------------------------

\begin{frame}
\begin{center}
\Huge{\hilit{Wrapping Up}}
\end{center}
\end{frame}

%------------------------------------------------

\begin{frame}
\frametitle{About XML}

\bb{About XML}
\bi
 \item designed to store and transfer data
 \item designed to be self-descriptive
 \item tags are not predefined and can be extended
\ei
\eb

\end{frame}

%------------------------------------------------

\begin{frame}
\frametitle{Characteristics of XML}

\bb{XML is}
\bi
 \item a generic language that provides structure and syntax for many markup dialects
 \item is a syntax or format for defining markup languages
 \item a standard for the semantic, hierarchical representation of data
 \item provides a general approach for representing all types of information  dialects
\ei
\eb

\end{frame}

%------------------------------------------------

\begin{frame}[fragile]
\frametitle{XML document example}

\bb{Simple XML}
\begin{verbatim}
<?xml version="1.0"?>
<!DOCTYPE movies>
<movie mins="126" lang="en">
  <!-- this is a comment -->
  <title>Good Will Hunting</title>
  <director>
    <first_name>Gus</first_name>
    <last_name>Van Sant</last_name>
  </director>
  <year>1998</year>
  <genre>drama</genre>
</movie>
\end{verbatim}
\eb

\end{frame}

%------------------------------------------------

\begin{frame}
\frametitle{XML Tree Structure}

\bb{Each element can have:}
\bi
 \item a Name
 \item any number of attributes
 \item optional content
 \item other nested elements
\ei
\eb

\bb{Traversing the tree}
There's a \textbf{unique} path from the root node to any given node 
\eb

\end{frame}

%------------------------------------------------

\begin{frame}
\begin{center}
\Huge{\hilit{XPath Language}}
\end{center}
\end{frame}

%------------------------------------------------

\begin{frame}[fragile]
\frametitle{XPath}

\bb{Querying Trees}
The real parsing power comes from the ability to \textbf{locate nodes and extract information from them}. For this, we need to be able to perform queries on the parsed content.
\eb

\bb{XPath}
The solution is provided by \textbf{XPath}, which is a language to navigate through elements and attributes in an XML/HTML document
\eb

\end{frame}

%------------------------------------------------

\begin{frame}
\frametitle{XPath}

\bb{XPath}
\bi
 \item is a language for finding information in an XML document
 \item uses path expressions to select nodes or node-sets in an XML document
 \item works by identifying patterns to match data or content
 \item includes over 100 built-in functions
\ei
\eb

\end{frame}

%------------------------------------------------

\begin{frame}
\frametitle{About XPath}

\bb{XPath Syntax}
XPath uses \textbf{path expressions} to select nodes in an XML document. It has a computational model to identify sets of nodes (node-sets)
\eb

\bb{XPath Syntax}
We can specify paths through the tree structure:
\bi
 \item based on node names
 \item based on node content
 \item based on a node's relationship to other nodes
\ei
\eb

\end{frame}

%------------------------------------------------

\begin{frame}[fragile]
\frametitle{About XPath}

\bb{XPath Syntax}
The key concept is knowing how to write XPath expressions. XPath expressions have a syntax similar to the way files are located in a hierarchy of directories/folders in a computer file system. For instance:
\eb

{\hilit \code{/movies/movie[1]}}

\bigskip
is the XPath expression to locate the first {\hilit \code{movie}} element that is the child of the {\hilit \code{movies}} element

\end{frame}

%------------------------------------------------

\begin{frame}
\frametitle{Selecting Nodes}

\bb{XPath Syntax}
The main path expressions (i.e. symbols) are:
\eb

\begin{center}
 \begin{tabular}{l l}
  \hline
  Symbol & Description \\
  \hline
  \code{/} & selects from the root node \\
  \code{//} & selects nodes anywhere \\
  \code{.} & selects the current node \\
  \code{..} & Selects the parent of the current node \\
  \code{@} & Selects attributes \\
  \code{[]} & Square brackets to indicate attributes \\
  \hline
 \end{tabular}
\end{center}

\end{frame}

%------------------------------------------------

\begin{frame}
XPath Predicates (square brackets \code{[]}) allow you to find a specific node or node\(s\) that contains a specific value.  You can use the usual comparison operators <, <=, etc.  A major difference is that equality is = and \textbf{NOT} ==.  An example usage might be:

\code{/store/gardening/plants/flowers[@avgheight>10]}

This would search the \textbf{store} document for flowers with an \textbf{attribute} \textbf{avgheight} greater than 10.
\end{frame}

%------------------------------------------------

\begin{frame}
\frametitle{Selecting Unknown Nodes}

\bb{XPath wildcards for unknown nodes}
XPath wildcards can be used to select unknown XML elements
\eb

\begin{center}
 \begin{tabular}{l l}
  \hline
  Symbol & Description \\
  \hline
  \code{*} & matches any element node \\
  \code{@*} & matches any attribute node \\
  \code{node()} & matches any node of any kind \\
  \hline
 \end{tabular}
\end{center}

\end{frame}

%------------------------------------------------

\begin{frame}
\frametitle{Movies Tree Structure}
\begin{center}
\ig[width=10cm]{images/xpath_tree.pdf}
\end{center}
\end{frame}

%------------------------------------------------

\begin{frame}
\frametitle{XPath: movie nodes}
\begin{center}
\ig[width=10cm]{images/xpath_movie.pdf}
\end{center}
\end{frame}

%------------------------------------------------

\begin{frame}
\frametitle{XPath: movie title nodes}
\begin{center}
\ig[width=10cm]{images/xpath_title.pdf}
\end{center}
\end{frame}

%------------------------------------------------

\begin{frame}
\frametitle{XPath: movie director's first name nodes}
\begin{center}
\ig[width=10cm]{images/xpath_firstname.pdf}
\end{center}
\end{frame}

%------------------------------------------------

\begin{frame}
\frametitle{XPath: movie director nodes}
\begin{center}
\ig[width=10cm]{images/xpath_director.pdf}
\end{center}
\end{frame}

%------------------------------------------------

\begin{frame}
\frametitle{XPath: last name nodes}
\begin{center}
\ig[width=10cm]{images/xpath_lastname.pdf}
\end{center}
\end{frame}

%------------------------------------------------

\begin{frame}
\frametitle{XPath: title node of movie in Spanish}
\begin{center}
\ig[width=10cm]{images/xpath_ytmt.pdf}
\end{center}
\end{frame}

%------------------------------------------------

\begin{frame}
\begin{center}
\Huge{\hilit{Querying parsed documents}}
\end{center}
\end{frame}

%------------------------------------------------

\begin{frame}[fragile]
\frametitle{XPath in "XML"}

\bb{XPath in \code{"xml2"}}
To work with XPath expressions using the \code{"xml2"} package, we have the auxiliary function {\hilit \code{xml\char`_find\char`_all()}} that accepts XPath expressions in order to 
select node-sets. Its main usage is:
\begin{verbatim}
    xml_find_all(xml_doc, xpath)
\end{verbatim}
\eb

where {\hilit \code{doc}} is an object of class \code{"xml\char`_document"} and {\hilit \code{xpath}} is a string giving the XPath expression to be evaluated

\end{frame}

%------------------------------------------------

\begin{frame}
\frametitle{Some References}

\bi
 \item XML Files website {\scriptsize (\url{http://www.xmlfiles.com})} \\
 {\lolit by Jan Egil Refsnes}
 \item XML in a  Nutshell \\
 {\lolit by Elliotte Rusty Harold; W. Scott Means}
 \item XML Tutorial {\scriptsize (\url{http://www.w3schools.com/xml/default.asp})} \\
 {\lolit by w3schools}
 \item Introduction to Data Technologies \\
 {\lolit by Paul Murrell}
 \item XML and Web Technologies for Data Sciences with R \\
 {\lolit by Deb Nolan and Duncan Temple Lang}
 \item xml2 {\scriptsize (\url{https://github.com/hadley/xml2})} \\
 {\lolit by Hadley Wickham}
\ei

\end{frame}

%------------------------------------------------


\end{document}
